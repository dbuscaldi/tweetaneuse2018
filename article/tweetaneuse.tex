%% Exemple de source LaTeX pour un article soumis à TALN 2017
\documentclass[10pt,twoside]{article}

\usepackage{times}
\usepackage[utf8]{inputenc}
\usepackage[T1]{fontenc}
\usepackage{graphicx}

% faire les \usepackage dont vous avez besoin AVANT le \usepackage{jeptaln2016} 

\usepackage{taln2017}
\usepackage[frenchb]{babel}

% Titre complet
\title{Modèles en Caractères pour la Détection de Polarité dans les Tweets}

\author{Davide Buscaldi, %Aude Grezka
 Joseph Le Roux\up{1}\quad Gaël Lejeune\up{2}\\
  {\small
    (1) LIPN, Université Paris XIII, 99 avenue JB Clément, 93430, Villetaneuse \\ 
    (2) STIH, Sorbonne Université, 28 rue Serpente, 75006, Paris \\ 
    \texttt{
      prenom.nom@lipn.univ-paris13.fr, prenom.nom@sorbonne-universite.fr \\ 
}}}

\begin{document}
\maketitle

\resume{

 Dans cet article, nous présentons notre contribution au Defi Fouille de Textes 2018 au travers de trois méthodes originales pour la classification thématique et la détection de polarité dans des \textit{tweets} en français. Nous y avons ajouté un système de vote.
 Notre première méthode est fondée sur des lexiques (mots et emojis) tandis que les deux autres sont des méthodes endogènes fondées sur des analyses au grain caractères : un modèle à mémoire à court-terme persistante (ou \textit{Bi-LSTM} pour \textit{Bidirectionnal Long Short-Term Memory}) d'une part et un modèle de séquences de caractères fermées fréquentes.
 Le \textit{Bi-LSTM} a produit de loin les meilleurs résultats puisqu'il a obtenu la première place sur la tâche 1, classification binaire de \textit{tweets} selon qu'ils traitent ou non des Transports, et la troisième place sur la tâche 2, classification de la polarité en 4 classes.
 Ce résultat est d'autant plus intéressant que la méthode proposée est faiblement paramétrique et qu'elle est totalement endogène. 

}

\abstract{Character-level Models for Polarity Detection in Tweets}{

We present our contribution to the DEFT 2018 shared task, with three entries based on different methods to perform topic classification and polarity detection for tweets in French, to which we added a voting system.
Our first entry is based on lexicons (for words and emojis), character n-grams and a classifier implemented with a support vector machine (\textit{SVM}), while the other two are endogenous methods based on character-level feature extraction: first a long short-memory recurrent neural network (\textit{Bi-LSTM}) feeding a classifier implementing a multi-layer perceptron, and second a model based on frequent closed character sequences with a {SVM}.
The \textit{Bi-LSTM} system gave the best results by far.
It ranked first on task 1, a binary theme classification task, and third on task 2, a four-class polarity classification task.
This result is very encouraging as this method has very few priors, is completely endogenous, and does not require any specific preprocessing.


%%% Local Variables:
%%% mode: latex
%%% TeX-master: "../tweetaneuse"
%%% End:

}

\motsClefs
  {Analyse en Caractères, Bi-LSTM, n-grammes de caractères, Détection de Polarité, Analyse de Tweets}
  {Character-level Models, Bi-LSTM, character n-grams, Polarity detection, Tweet Analysis}


\section{Introduction}


L'analyse automatique des \textit{tweets} est un domaine très actif du Traitement Automatique des Langues (TAL), notamment sur l'aspect étude de la polarité et détection des émotions.
Ceci n'est pas très étonnant dans la mesure où les \textit{tweets}, et plus généralement les microblogs, constituent un moyen de jauger l'opinion individuelle d'une grande quantité de personnes.
Il existe donc un grand nombre d'applications qui pourraient profiter de systèmes performants pour cette tâche.
D'un autre côté, la classification binaire de \textit{tweets} selon qu'ils relèvent ou non d'une thématique particulière est une tâche connexe à la détection des émotions mais souvent délaissée, et placée en amont de la chaîne de traitement de {TAL}.
Elle est traditionnellement implémentée par filtrage sur des chaînes de caractères ou des expressions rationnelles qui font référence à des termes du thème visé.
C'est regrettable car connaître la polarité d'une émotion est intéressant mais ça l'est plus encore si l'on sait à quoi le \textit{tweet} fait référence.

%Dans cette édition du DEFT, et contrairement à l'édition 2015 par exemple, les deux tâches mentionnées ci-dessus  ont été séparées.
D'un côté la tâche 1 consiste donc à décider si un \textit{tweet} traite ou non des transports et de l'autre côté la tâche 2 pour laquelle, étant donné un \textit{tweet} traitant des transports, il faut détecter la polarité: Neutre, Négatif, Positif ou Mixte.
L'article décrivant cette édition du Défi Fouille de Textes propose une description précise du processus de collecte et d'annotation du corpus~\cite{Paroubek-2018}, nous ne reviendrons donc pas plus ici sur les données d'entrée.
% Nous nous contenterons ici d'une présentation rapide du contenu des jeux de données d'apprentissage et de test et de l'influence que ceci a eu sur le développement de nos approches.

% Dans la section \ref{sec:corpus} nous décrirons un peu plus précisément le jeu de données puis
Dans la section~\ref{sec:methodes} nous décrirons les trois méthodes que nous avons mises en place ainsi qu'un système de vote entre les méthodes.
Ensuite, nous présenterons dans la section~\ref{sec:resultats} les résultats que nous avons obtenus dans la compétition officielle ainsi que des expériences plus complètes sur nos méthodes.
Enfin, nous proposerons quelques conclusions et perspectives sur ce travail dans la section~\ref{sec:conclusion}.

%%% Local Variables:
%%% mode: latex
%%% TeX-master: "../tweetaneuse"
%%% End:

\section{Corpus}

\label{sec:corpus} 

 L'article décrivant cette édition défi présenté une description du processus de collecte et d'annotation du corpus \cite{Paroubek-2018}.
 Nous nous contenterons ici d'une présentation rapide du contenu des jeux de données d'apprentissage et de test et de l'influence que ceci a eu sur le développement de nos approches.
.


\section{Méthodes}

\subsection{Méthode 1}

Le premier système est basé sur des SVM avec un noyau gaussien, où les vecteurs comportent un mélange de caractéristiques à niveau de caractères et lexicales, à l'aide de dictionnaires.

Pour les caractères, nous avons choisi d'utiliser, en tant que caractéristiques, tous les n-grammes de caractères (espaces exclues) de taille comprise entre 3 et 6, avec une fréquence minimale de 100 (arbitrairement choisie) dans le corpus d’entraînement.
Ce choix est fondé sur l'observation que les n-grammes de caractères permettent de capturer des variations sur un même mot ou acronyme qui sont fréquentes dans le langage des \textit{tweets}, par exemple: \emph{\#ratp}, \emph{@ligne7\_ratp}, \emph{\#merciratp}, \emph{@grouperatp}, etc.
Le poids d'un n-gramme est de 0 s'il n'apparaît pas dans le \textit{tweet} ou $s(n)= \sum_{i=1}^{nbOcc} 1 + pos(n_i)/len(t)$ s'il apparaît dans le \textit{tweet}, où $pos(n_i)$ indique la position du premier caractère de l'$i$-ème occurrence du n-gramme $n$ et $len(t)$ est la taille du \textit{tweet} en nombre de caractères.

Nous avons choisi de garder aussi dans le modèle une composante lexicale, avec l'utilisation de dictionnaires de polarité, en particulier les dictionnaires labMT \cite{dodds2011} et FEEL \cite{abdaoui2017}.
Les dictionnaires ont été utilisés pour associer deux poids à chaque \textit{tweet}~: un poids pour la première moitié du \textit{tweet} et un autre poids pour la deuxième moitié.
Les scores des dictionnaires ont été normalisés dans l'intervalle $[-1, +1]$, dans le cas du FEEL cela revient à associer le label \emph{négatif} à $-1$ et \emph{positif} à $+1$.
Pour labMT, étant donné que les scores sont dans l'intervalle $(1,9)$, on a associé le minimum à $-1$ et le maximum à $+1$ puis on a mis à l'échelle les valeurs intermédiaires. Finalement pour calculer la valeur des caractéristiques pour chaque moitié du \textit{tweet} on garde la somme des scores des mots.

Pour la tâche T2, le système a utilisé les résultats de la tâche T1 pour travailler uniquement sur les \textit{tweets} étiquetés \emph{Tranport}.
On a donc créé un modèle de polarité pour détecter si les \textit{tweets} avaient une polarité ou pas.
La sortie de ce modèle a été utilisé comme entrée pour un modèle dédié à la polarité positive et un modèle dédié à la polarité négative.
Finalement nous avons produit un script pour combiner les résultats produits par chaque modèle.
Donc si un \textit{tweet} avait été reconnu comme \emph{Transport}, alors on vérifie s'il est polarisé ou pas.
S'il est polarisé, on vérifie s'il est \emph{positif} ou pas, on vérifie s'il est \emph{négatif} ou pas, et s'il est \emph{positif et négatif} en même temps on lui assigne l'étiquette \emph{MIXPOSNEG}.

%%% Local Variables:
%%% mode: latex
%%% TeX-master: "../tweetaneuse"
%%% End:

\subsection{Méthode 2: BiLSTM}
Notre deuxième système utilise des réseaux de neurones récurrents pour implanter un classificateur, plus spécifiquement les LSTM~\cite{hochreiter1997long} qui sont largement utilisés en traitement automatiques des langues.
La classification se fait en trois temps~:
\begin{enumerate}
\item Le texte est séparé aux espaces.
  Chaque segment est traité comme une séquence d'octets lue de gauche à droite et de droite à gauche par deux réseaux récurrents \emph{niveau caractère}.
  Les vecteurs résultats des lectures sont additionnés et servent de représentation du segment, dite compositionnelle.
  Pour une séquence de caractères $s = c_{1}\ldots c_{m}$, on calcule pour chaque position $h_{i} = LSTM_{o}(h_{i-1},e(c_{i}))$ et $h'_{i} = LSTM_{o'}(h'_{i+1},e(c_{i}))$, où $e$ est la fonction de plongement des caractères vers les vecteurs denses, et $LSTM$ est un raccourci pour une fonction  implantant la cellule récurrente des LSTM.
  La représentation compositonnelle du segment est $c(s) = h_{m} + h'_{1}$

\item La séquence de segments est lue à nouveau de gauche à droite et de droite à gauche par de nouveaux réseaux récurrents \emph{niveau mot} qui prennent en entrée pour chaque segment la représentation compositionnelle venant de l'étape précédente à laquelle on ajoute une représentation vectorielle du segment si celui-ci était présent  plus de 10 fois dans le corpus d'entraînement.
  Pour une séquence de segments $p = s_{1} \ldots s_{n}$, on calcule
  $l_{i} = LSTM_{m}(l_{i-1},c(s_{i}) + e(s_{i}))$, $l'_{i} =
  LSTM_{m'}(l_{i+1},c(s_{i}) + e(s_{i}))$, où $c$ est la représentation compositionnelle donnée ci-dessus et $e$ une fonction de plongement du segment.

  Les états finaux obtenues après lecture dans les deux directions sont sommés et servent de représentation de la phrase d'entrée, $r(p) = l_{n} + l'_{1}$.

\item La représentation obtenue sert d'entrée à un perceptron multi-niveau qui effectue la classification finale, aussi bien pour la tâche 1 que pour la tâche 2:  $o(p) = \sigma(O \times \max(0, (W \times r(p) + b)))$ où $\sigma$ est l'opérateur \emph{softmax}, $W$, $O$ des matrices et $b$ un vecteur.
\end{enumerate}

L'apprentissage du système se fait en maximisant la vraisemblance du corpus d'entraînement.
On utilise l'algorithme AMSgrad pour  calculer la taille du pas lors de la descente de gradient.
On écarte aléatoirement du corpus d'entraînement 10\% des phrases qui sont utilisées comme ensemble de validation.
Les plongements de caractères sont de taille 16, ceux des segments de taille 32, la couche d'entrée du perceptron est de taille 64 et la couche cachée de taille 32.
La couche de sortie est de taille 2 pour la tâche 1, et 4 pour la tâche 2.
Pour éviter le sur-apprentissage nous utilisons la technique du \emph{dropout} \cite{srivastava2014dropout}, sur tous les vecteurs à chaque étage du réseau (sauf la couche finale bien sûr)~: durant l'apprentissage les neurones sont aléatoirement mis à zéro avec une probabilité de 0,5.


%%% Local Variables:
%%% mode: latex
%%% TeX-master: "../tweetaneuse"
%%% End:

\subsection{Méthode 3: motifs en caractères}
Notre troisième système utilise également une approche d'analyse au grain caractère.
Plus exactement cette approche se situe à la lisière entre l'algorithmique du texte et la fouille de données.
Nous utilisons des motifs (en caractères) fermés et fréquents comme traits pour entraîner un classificateur.

Les propriétés de fermeture et de fréquence sont définis de la façon suivante :%\footnote{Il s'agit d'une transposition de \textit{frequency}, il s'agit donc d'effectif}
\begin{description}
\item[Fermeture~:] le motif ne peut être étendu vers la gauche ou vers la droite sans diminuer son nombre d'occurrences
\item[Fréquence~:] le motif respecte une borne minimale de nombre d'apparitions
\end{description}

Pour calculer des motifs en caractères de manière efficace, en l'occurrence avec une complexité linéaire en la taille des données, nous utilisons ici une implantation en Python de l'algorithme décrit dans~\cite{Ukkonen-2009} exploitant les tableaux de suffixes augmentés décrits dans~\cite{Karkka-2006}.
\cite{Buscaldi-2017} rappellent que les propriétés de fermeture et de fréquence correspondent en algorithmique du texte aux propriétés de maximalité et de répétition.
Du point de vue du TAL, cette technique peut être décrite comme une segmentation \emph{non-supervisée} en ce sens que les règles de découpage ne sont pas pré-définies mais sont calculées en fonction du corpus donné en entrée. %  Les motifs d'effectif 1 ne sont pas exploités

De façon traditionnelle en fouille de données, un défi important est de limiter l'explosion du nombre de motifs, que ce soit pour des raisons calculatoires ou pour des raisons de lisibilité des résultats.% En effet, nous pouvons voir dans le tableau \ref{tab:nbMotifs} que le nombre de motifs peut être très important.
Un filtrage classique consiste à appliquer aux motifs deux types de contrainte:
\begin{itemize}
\item La contrainte de support par laquelle on définit le nombre minimal ($minsup$) et maximal ($maxsup$) d'objets qui supportent un motif.
  Ici cela consiste à définir le nombre minimal et maximal de \textit{tweets} dans lequel le motif apparaît.
\item La contrainte de longueur par laquelle on définit la longueur minimale ($minlen$) et maximale ($maxlen$) d'un motif.
  Ici, il s'agit d'une longueur en caractères.
\end{itemize}

Notre chaîne de traitement est définie comme suit:
\begin{enumerate}
  \item Calcul des motifs fermés fréquents dans tout le corpus de \textit{tweets} (train+test);
  \item Filtrage des motifs selon la longueur;
  \item Filtrage des motifs selon le support;
  \item Représentation de chaque \textit{tweet} sous forme d'un vecteur d'effectif des motifs;
  \item Utilisation de l'implantation de \textsc{SciKit} du SVM \textsc{One vs Rest}.
\end{enumerate}


Durant la phase d'entraînement nous avons testé la méthode au travers d'un validation croisée en 10 strates au moyen de la fonction \textsc{StratifiedKFold} de \textsc{SciKit}\footnote{\url{http://scikit-learn.org/}}.
 En examinant les résultats, nous avons observé que la qualité des résultats n'augmentait plus au delà d'un seuil de $maxlen$ de 5.
Nous avons également remarqué, ce qui est conforme à des résultats précédents sur des données comparables~\cite{Buscaldi-2017} que l'application de contraintes de support avait assez peu d'influence sur les résultats.
Nous avons donc choisi de ne pas chercher à paramétrer finement cette contrainte.
Toutefois, il est à noter que la réduction de l'espace de description engendrée par cette contrainte permet de diminuer le coût en calcul.

Les résultats que nous avons soumis pour la phase de test ont été obtenus avec les configurations suivantes~:
\begin{itemize}
  \item Pas de taille minimale ($minlen=1$);
  \item Pour la tâche 1 $maxlen=2$ et pour la tâche 2 $maxlen=3$;
  \item Pas de contrainte de support;
  \item Utilisation d'un noyau linéaire.
\end{itemize}

La configuration ci-dessus s'est avérée la plus efficace sur les données d'entraînement et la plus fiable pour effectuer les calculs dans le temps imparti.
L'utilisation d'un noyau radial permettait toutefois d'obtenir de bons résultats y compris en se contentant des motifs de taille 1 ($maxlen=1$).



%%% Local Variables:
%%% mode: latex
%%% TeX-master: "../tweetaneuse"
%%% End:

\subsection{Système de vote}
 Avec notre quatrième run, nous avons tenté de tirer profit des propriétés respectives de nos trois méthodes en élaborant un système de vote.
 Pour les deux tâches, il s'agit simplement d'un vote majoritaire entre les résultats des trois méthodes.
Pour la tâche 2 qui comporte 4 classes, il a fallu gérer les cas d'égalité qui représentaient près de 25\% des cas.
 Nous avons considéré que le fait que l'absence d'accord tangible entre les méthodes indiquait que nous avions à faire à des tweets sans tendance particulière et nous avons donc donné l'étiquette "MIXPOSNEG". Les cas d'indécision étant plutôt fréquents, cela a abouti à une sur-représentation de cette classe qui a nui aux résultats comme nous le montrerons dans la section suivante.



\section{Résultats}
\label{sec:resultats}

Dans la tâche 1, nos

\begin{table}[h]
\begin{tabular}{l|l|l|l|l|l|l|l}
Run		& Précision	& Rappel& F-mesure	& VP 	&FP 	&FN\\
\hline
\hline
Run1 (Lexique et caractères)  	&0.80463   	& 1.0	& 0.89174 	&6289	&1527	&0\\
Run2 (BiLSTM)	& 0.831246497 	& 1.0	& 0.90785	&6497	&1319	&0\\
Run3 (motifs en caractères)	&0.77364	& 0.999 & 0.87231	&6046	&1769	&1\\
Run4 (vote)&0.824826446  & 0.999	& 0.90394	&6446	&1369	&1\\
\hline

\end{tabular}
\caption{Résultats officiels de Tweetaneuse sur la tâche 1\label{tab:resultats_T1}}
\end{table}



Dans la tâche 2, l'écart entre nos système s'est creusé.
1033 accords complets
1391 2 systèmes ont la bonne réponse et influencent en conséquence le système de vote
Dans 39 cas le système de vote a résolu une indécision avec une étiquette MIXPOSNEG à raison

\begin{table}
\begin{tabular}{l|l|l|l|l|l|l|l}
Run		& Précision	& Rappel& F-mesure	& VP 	&FP 	&FN\\
\hline
\hline
Run1 (Lexique et caractères)   	&0.47599 	&0.93313 	&0.63041	&1814	&1997	&130\\
Run2 (BiLSTM) 	&0.67699	&1		&0.80738	&2668	&1273	&0\\
Run3 (Motifs en caractères) 	&0.57828&1  		&0.7328 	&2279	&1662	&0\\
Run4 (vote) 	&0.62766  	&0.9996  	&0.77113 	&2473	&1467	&1\\
\hline


\end{tabular}
\caption{Résultats officiels de Tweetaneuse sur la tâche 2\label{tab:resultats_T2}}
\end{table}



\section{Discussion}
\label{sec:conclusion}
Dans cet article, nous avons présenté trois méthodes exploitant le grain caractère pour la classification de \textit{tweets} et la détection de polarité.
Ces trois méthodes utilisent de l'apprentissage supervisé.
La première méthode combine ressources exogènes, des lexiques, et n-grammes de caractères, la seconde exploite un Bi-LSTM tandis que la troisième utilise des motifs en caractères fermés fréquents.

Les bons résultats obtenus, en particulier pour le Bi-LSTM montrent que l'utilisation du grain caractère pour les tâches de TAL est amenée à avoir un impact important.
En effet, les méthodes en caractère ont l'avantage de bien se comporter dans des contextes bruités, ce qui est le cas ici puisque les variations de graphie et la distanciation avec les normes syntaxiques sont des caractéristiques majeures des \textit{tweets}.
Par ailleurs, ces approches permettent de se passer de pré-traitement ce qui permet de simplifier les chaînes de traitement et par conséquent de réduire les erreurs en cascade\cite{Lejeune-2014}.
Enfin, ces méthodes permettent de capturer de manière purement endogène des informations sur la structure intra-token (morphèmes) comme sur la structure extra-token (expressions multi-mots par exemple) ce qui peut être particulièrement intéressant dans un contexte impliquant plusieurs langues ou des langues peu dotées en ressources.
Nos prototypes sont disponibles en ligne\footnote{\url{https://github.com/rcln/tweetaneuse2018}} et librement utilisables.


%%% Local Variables:
%%% mode: latex
%%% TeX-master: "../tweetaneuse"
%%% End:




\bibliographystyle{taln2017}
\bibliography{biblio}
%% commande no-cite pour inclure dans la bibliographie des références 
%% qui ne sont pas citées dans le corps de l'article

%%================================================================
\end{document}
