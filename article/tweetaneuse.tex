%% Exemple de source LaTeX pour un article soumis à TALN 2017
\documentclass[10pt,twoside]{article}

\usepackage{times}
\usepackage[utf8]{inputenc}
\usepackage[T1]{fontenc}
\usepackage{graphicx}

% faire les \usepackage dont vous avez besoin AVANT le \usepackage{jeptaln2016} 

\usepackage{taln2017}
\usepackage[frenchb]{babel}

% Titre complet
\title{Modèles en Caractères\\ pour la Détection de Polarité dans les Tweets}

\author{Davide Buscaldi\up{1}\quad Joseph Le Roux\up{1}\quad Gaël Lejeune\up{2}\\
  {\small
    (1) LIPN, Université Paris XIII, 99 avenue JB Clément, 93430, Villetaneuse \\ 
    (2) STIH, Sorbonne Université, 28 rue Serpente, 75006, Paris \\ 
    \texttt{
      prenom.nom@lipn.univ-paris13.fr, prenom.nom@sorbonne-universite.fr \\ 
}}}

\begin{document}
\maketitle

\resume{

 Dans cet article, nous présentons notre contribution au Défi Fouille de Textes 2018 au travers de trois méthodes originales pour la classification thématique et la détection de polarité dans des \textit{tweets} en français. Nous y avons ajouté un système de vote.
 Notre première méthode est fondée sur des lexiques (mots et emojis) et un classificateur à vaste marge
 tandis que les deux autres sont des méthodes endogènes fondées sur l'extraction de caractéristiques au grain caractères~: un modèle à mémoire à court-terme persistante (ou \textit{Bi-LSTM} pour \textit{Bidirectionnal Long Short-Term Memory}) et perceptron multi-couche d'une part et un modèle de séquences de caractères fermées fréquentes et classificateur à large marge d'autre part.
 Le \textit{Bi-LSTM} a produit de loin les meilleurs résultats puisqu'il a obtenu la première place sur la tâche 1, classification binaire de \textit{tweets} selon qu'ils traitent ou non des transports, et la troisième place sur la tâche 2, classification de la polarité en 4 classes.
 Ce résultat est d'autant plus intéressant que la méthode proposée est faiblement paramétrique et qu'elle est totalement endogène et n'implique aucun prétraitement.

%%% Local Variables:
%%% mode: latex
%%% TeX-master: "../tweetaneuse"
%%% End:

}

\abstract{Character-level Models for Polarity Detection in Tweets}{

We present our contribution to the DEFT 2018 shared task, with three entries based on different methods to perform topic classification and polarity detection for tweets in French, to which we added a voting system.
Our first entry is based on lexicons (for words and emojis), character n-grams and a classifier implemented with a support vector machine (\textit{SVM}), while the other two are endogenous methods based on character-level feature extraction: first a long short-memory recurrent neural network (\textit{Bi-LSTM}) feeding a classifier implementing a multi-layer perceptron, and second a model based on frequent closed character sequences with a {SVM}.
The \textit{Bi-LSTM} system gave the best results by far.
It ranked first on task 1, a binary theme classification task, and third on task2, a four-class polarity classification task.
This result is very encouraging as this method has very few priors, is completely endogenous, and does not require any specific preprocessing.


%%% Local Variables:
%%% mode: latex
%%% TeX-master: "../tweetaneuse"
%%% End:

}

\motsClefs
  {Analyse en Caractères, Bi-LSTM, n-grammes de caractères, Détection de Polarité, Analyse de Tweets}
  {Character-level Models, Bi-LSTM, character n-grams, Polarity detection, Tweet Analysis}


\section{Introduction}


 Dans cet article,

%\section{Corpus}
%
 Dans cet article,


\section{Méthodes~: trois modèles en caractères et un système de vote}
\label{sec:methodes}

\subsection{Méthode 1}
 Notre premier run


\subsection{Méthode 2: BiLSTM}
Notre deuxième système utilise des réseaux de neurones récurrents pour implanter un classificateur, plus spécifiquement les LSTM~\cite{hochreiter1997long} qui sont largement utilisés en traitement automatiques des langues.
La classification se fait en trois temps~:
\begin{enumerate}
\item Le texte est séparé aux espaces.
  Chaque segment est traité comme une séquence d'octets lue de gauche à droite et de droite à gauche par deux réseaux récurrents \emph{niveau caractère}.
  Les vecteurs résultats des lectures sont additionnés et servent de représentation du segment, dite compositionnelle.
  Pour une séquence de caractères $s = c_{1}\ldots c_{m}$, on calcule pour chaque position $h_{i} = LSTM_{o}(h_{i-1},e(c_{i}))$ et $h'_{i} = LSTM_{o'}(h'_{i+1},e(c_{i}))$, où $e$ est la fonction de plongement des caractères vers les vecteurs denses, et $LSTM$ est un raccourci pour une fonction  implantant la cellule récurrente des LSTM.
  La représentation compositonnelle du segment est $c(s) = h_{m} + h'_{1}$

\item La séquence de segments est lue à nouveau de gauche à droite et de droite à gauche par de nouveaux réseaux récurrents \emph{niveau mot} qui prennent en entrée pour chaque segment la représentation compositionnelle venant de l'étape précédente à laquelle on ajoute une représentation vectorielle du segment si celui-ci était présent  plus de 10 fois dans le corpus d'entraînement.
  Pour une séquence de segments $p = s_{1} \ldots s_{n}$, on calcule
  $l_{i} = LSTM_{m}(l_{i-1},c(s_{i}) + e(s_{i}))$, $l'_{i} =
  LSTM_{m'}(l_{i+1},c(s_{i}) + e(s_{i}))$, où $c$ est la représentation compositionnelle donnée ci-dessus et $e$ une fonction de plongement du segment.

  Les états finaux obtenues après lecture dans les deux directions sont sommés et servent de représentation de la phrase d'entrée, $r(p) = l_{n} + l'_{1}$.

\item La représentation obtenue sert d'entrée à un perceptron multi-niveau qui effectue la classification finale, aussi bien pour la tâche 1 que pour la tâche 2:  $o(p) = \sigma(O \times \max(0, (W \times r(p) + b)))$ où $\sigma$ est l'opérateur \emph{softmax}, $W$, $O$ des matrices et $b$ un vecteur.
\end{enumerate}

L'apprentissage du système se fait en maximisant la vraisemblance du corpus d'entraînement.
On utilise l'algorithme AMSgrad pour  calculer la taille du pas lors de la descente de gradient.
On écarte aléatoirement du corpus d'entraînement 10\% des phrases qui sont utilisées comme ensemble de validation.
Les plongements de caractères sont de taille 16, ceux des segments de taille 32, la couche d'entrée du perceptron est de taille 64 et la couche cachée de taille 32.
La couche de sortie est de taille 2 pour la tâche 1, et 4 pour la tâche 2.
Pour éviter le sur-apprentissage nous utilisons la technique du \emph{dropout} \cite{srivastava2014dropout}, sur tous les vecteurs à chaque étage du réseau (sauf la couche finale bien sûr)~: durant l'apprentissage les neurones sont aléatoirement mis à zéro avec une probabilité de 0,5.


%%% Local Variables:
%%% mode: latex
%%% TeX-master: "../tweetaneuse"
%%% End:


\subsection{Méthode 3: motifs en caractères}
 Notre troisième système utilise également une approche d'analyse au grain caractère. Plus exactement cette approche se situe à la lisière en l'algorithmique du texte et la fouille de données.
 Nous utilisons des motifs (en caractères) fermés et fréquents comme traits pour entraîner un classifieur.

 Les propriétés de fermeture et de fréquence sont définis de la façon suivante :%\footnote{Il s'agit d'une transposition de \textit{frequency}, il s'agit donc d'effectif}
\begin{description}
\item[Fermeture]: le motif ne peut être étendu vers la gauche ou vers la droite sans diminuer son nombre d'occurences
\item[Fréquence]: le motif respecte une borne minimale de nombre d'apparitions ou support minimal
\end{description}

 Pour calculer des motifs en caractères de manière efficace, en l'occurrence avec une complexité linéaire en la taille des données, nous utilisons ici une implantation en Python de l'algorithme de Ukkonen \cite{Ukkonen-2009} exploitant les tableaux de suffixes augmentés décrits par Juha Kärkkainen (\cite{Karkka-2006}).
 \cite{Buscaldi-2017} rappelent que les propriétés de fermeture et de fréquence correspondent en algorithmique du texte aux propriétés de maximalité et de répétition.
 Cette technique amène à une tokenization que nous qualifierons de "non-supervisée" en ce sens que les règles de découpage ne sont pas pré-définies mais sont calculées en fonction du corpus donné en entrée.%  Les motifs d'effectif 1 ne sont pas exploités
 
 De façon traditionnelle en fouille de données, un défi important est de limiter l'explosion du nombre de motifs, que ce soit pour des raisons calculatoires ou pour des raisons de lisibilité des résultats.% En effet, nous pouvons voir dans le tableau \ref{tab:nbMotifs} que le nombre de motifs peut être très important.
 Un filtrage classique consiste à appliquer aux motifs deux types de contraintes:
\begin{itemize}
  \item La contrainte de support (qui vient en quelque sorte affiner la propriété de fréquence) par laquelle on définit le nombre minimal ($minsup$) et maximal ($maxsup$) d'objets qui supportent un motif. Ici cela consiste à définir le nombre minimal et maximal de tweets dans lequel le motif apparaît.
  \item La contrainte de longueur par laquelle on définit la longueur minimale ($minlen$) et maximale ($maxlen$) d'un motif. Ici, il s'agit d'une longueur en caractères.
\end{itemize}

 Notre chaîne de traitement est donc définie comme suit:
\begin{enumerate}
  \item Calcul des motifs fermés fréquents dans tout le corpus
  \item Filtrage selon des critères de longueur et de support
  \item Représentation de chaque tweet sous forme d'un vecteur d'effectif
  \item Utilisation de l'implantation dans \textsc{SciKit} du SVM \textsc{One VS Rest}
\end{enumerate}

 
Durant la phase d'entraînement nous testé la méthode au travers d'un validation croisée en 10 strates au moyen de la fonction StratifiedKFold de \textsc{SciKit}. En examinant les résultats, nous avons observé que la qualité des résultats n'augmentait plus au delà d'un seuil de $maxlen$ de 5.
Nous avons également remarqué, ce qui est conforme à des résultats précédents sur des données comparables (\cite{Buscaldi-2017}), que l'application de contraintes de support avait assez peu d'influence sur les résultats. Nous avons donc choisi de ne pas chercher à paramétrer finement cette contrainte. Toutefois, dans une perspective calculatoire, la réduction de l'espace de description engendrée par cette contrainte aurait du sens.
 


\subsection{Système de vote}
 Avec notre quatrième run, nous avons tenté de tirer profit des propriétés respectives de nos trois méthodes en élaborant un système de vote.
 Pour les deux tâches, il s'agit simplement d'un vote majoritaire entre les résultats des trois méthodes.
Pour la tâche 2 qui comporte 4 classes, il a fallu gérer les cas d'égalité qui représentaient près de 25\% des cas.
 Nous avons considéré que le fait que l'absence d'accord tangible entre les méthodes indiquait que nous avions à faire à des tweets sans tendance particulière et nous avons donc donné l'étiquette "MIXPOSNEG". Les cas d'indécision étant plutôt fréquents, cela a abouti à une sur-représentation de cette classe qui a nui aux résultats comme nous le montrerons dans la section suivante.



\section{Résultats}

 Dans cet article,


\section{Discussion}

 Dans cet article,




\bibliographystyle{taln2017}
\bibliography{biblio}
%% commande no-cite pour inclure dans la bibliographie des références 
%% qui ne sont pas citées dans le corps de l'article

%%================================================================
\end{document}
