 Avec notre quatrième run, nous avons tenté de tirer profit des propriétés respectives de nos trois méthodes en élaborant un système de vote.
 Pour les deux tâches, il s'agit simplement d'un vote majoritaire entre les résultats des trois méthodes.
Pour la tâche 2 qui comporte 4 classes, il a fallu gérer les cas d'égalité qui représentaient près de 25\% des cas.
 Nous avons considéré que l'absence d'accord tangible entre les méthodes indiquait que nous avions à faire à des \textit{tweets} sans tendance particulière. Nous avons donc donné l'étiquette \emph{MIXPOSNEG}. Les cas d'indécision étant plutôt fréquents, cela a abouti à une sur-représentation de cette classe qui a nui aux résultats comme nous le montrerons dans la section suivante.
