Le premier système est basé sur des machines à vecteurs de support avec un noyau gaussien, où les vecteurs comportent un mélange de caractéristiques à niveau de caractères et lexicales, basées sur des dictionnaires.

Pour les caractères, nous avons choisi d'utiliser, en tant que caractéristiques, tous les n-grammes de caractères (espaces exclus) de taille comprise entre 3 et 6, avec une fréquence minimale de 100 (arbitrairement choisie) dans le corpus d’entraînement. Ce choix est fondé sur l'observation que les n-grammes
de caractères permettent de capturer des variations sur un même mot ou acronyme qui sont fréquentes dans le langage des tweets, par exemple: \emph{\#ratp}, \emph{@ligne7\_ratp}, \emph{\#merciratp}, \emph{@grouperatp}, etc. Le poids d'un n-gramme est calculé comme 0 s'il n'apparaît pas dans le tweet ou $s(n)= \sum_{i=1}^{nbOcc} 1 + pos(n_i)/len(t)$ s'il apparaît dans le tweet, où $pos(n_i)$ indique la position du premier caractère de l'$i$-ème occurrence du n-gramme $n$ et $len(t)$ est la taille du tweet en nombre de caractères.

Nous avons choisi de garder aussi dans le modèle une composante lexique, avec l'utilisation de dictionnaires de polarité, en particulier les dictionnaires labMT \cite{dodds2011} et FEEL \cite{abdaoui2017}. Les dictionnaires ont été utilisé pour associer à chaque tweet deux poids: un poids pour la première moitié du tweet et un autre poids pour la deuxième moitié. Les scores des dictionnaires ont été normalisés dans le range $[-1, +1]$: dans le cas du FEEL cela revient à associer le label ``négatif" à $-1$ et ``positif" à $+1$. 
Pour labMT, vu que les scores sont dans l'intervalle $(1,9)$, on a associé le minimum à $-1$ et le max à $+1$ et on a mis à l'échelle les valeurs intermédiaires. Finalement pour calculer la valeur des caractéristiques pour chaque moitié du tweet on garde la somme des scores des mots.

Pour la tâche T2, le système a utilisé les résultats de la tâche T1 pour travailler uniquement sur les tweets labellisé comme TRANS. On a donc créé un modèle de polarité pour détecter si les tweets avaient une polarité ou pas. La sortie de ce modèle a été utilisé comme entrée pour un modèle dédié à la polarité positive et un modèle dédié à la polarité négative. Finalement nous avons produit un script pour combiner les résultats produits par chaque modèle. Donc si un tweet avait été reconnu comme TRANS, alors on vérifie s'il est polarisé ou pas. S'il est polarisé, on vérifie s'il est positif ou pas, on vérifie s'il est négatif ou pas, et s'il est positif et négatif au même temps on assigne le label MIXPOSNEG.
