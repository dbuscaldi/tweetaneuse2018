
Le premier système est basé sur des SVM avec un noyau gaussien, où les vecteurs comportent un mélange de caractéristiques à niveau de caractères et lexicales, à l'aide de dictionnaires.

Pour les caractères, nous avons choisi d'utiliser, en tant que caractéristiques, tous les n-grammes de caractères (espaces exclues) de taille comprise entre 3 et 6, avec une fréquence minimale de 100 (arbitrairement choisie) dans le corpus d’entraînement.
Ce choix est fondé sur l'observation que les n-grammes de caractères permettent de capturer des variations sur un même mot ou acronyme qui sont fréquentes dans le langage des \textit{tweets}, par exemple: \emph{\#ratp}, \emph{@ligne7\_ratp}, \emph{\#merciratp}, \emph{@grouperatp}, etc.
Le poids d'un n-gramme est de 0 s'il n'apparaît pas dans le \textit{tweet} ou $s(n)= \sum_{i=1}^{nbOcc} 1 + pos(n_i)/len(t)$ s'il apparaît dans le \textit{tweet}, où $pos(n_i)$ indique la position du premier caractère de l'$i$-ème occurrence du n-gramme $n$ et $len(t)$ est la taille du \textit{tweet} en nombre de caractères.

Nous avons choisi de garder aussi dans le modèle une composante lexicale, avec l'utilisation de dictionnaires de polarité, en particulier les dictionnaires labMT \cite{dodds2011} et FEEL \cite{abdaoui2017}.
Les dictionnaires ont été utilisés pour associer deux poids à chaque \textit{tweet}~: un poids pour la première moitié du \textit{tweet} et un autre poids pour la deuxième moitié.
Les scores des dictionnaires ont été normalisés dans l'intervalle $[-1, +1]$, dans le cas du FEEL cela revient à associer le label \emph{négatif} à $-1$ et \emph{positif} à $+1$.
Pour labMT, étant donné que les scores sont dans l'intervalle $(1,9)$, on a associé le minimum à $-1$ et le maximum à $+1$ puis on a mis à l'échelle les valeurs intermédiaires. Finalement pour calculer la valeur des caractéristiques pour chaque moitié du \textit{tweet} on garde la somme des scores des mots.

Pour la tâche T2, le système a utilisé les résultats de la tâche T1 pour travailler uniquement sur les \textit{tweets} étiquetés \emph{Tranport}.
On a donc créé un modèle de polarité pour détecter si les \textit{tweets} avaient une polarité ou pas.
La sortie de ce modèle a été utilisé comme entrée pour un modèle dédié à la polarité positive et un modèle dédié à la polarité négative.
Finalement nous avons produit un script pour combiner les résultats produits par chaque modèle.
Donc si un \textit{tweet} avait été reconnu comme \emph{Transport}, alors on vérifie s'il est polarisé ou pas.
S'il est polarisé, on vérifie s'il est \emph{positif} ou pas, on vérifie s'il est \emph{négatif} ou pas, et s'il est \emph{positif et négatif} en même temps on lui assigne l'étiquette \emph{MIXPOSNEG}.

%%% Local Variables:
%%% mode: latex
%%% TeX-master: "../tweetaneuse"
%%% End:
