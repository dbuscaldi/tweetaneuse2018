
L'analyse automatique des \textit{tweets} est un domaine très actif du Traitement Automatique des Langues (TAL), notamment sur l'aspect étude de la polarité et détection des émotions.
Ceci n'est pas très étonnant dans la mesure où les \textit{tweets}, et plus généralement les microblogs, constituent un moyen de jauger l'opinion individuelle d'une grande quantité de personnes.
Il existe donc un grand nombre d'applications qui pourraient profiter de systèmes performants pour cette tâche.
D'un autre côté, la classification binaire de \textit{tweets} selon qu'ils relèvent ou non d'une thématique particulière est une tâche connexe à la détection des émotions mais souvent délaissée, et placée en amont de la chaîne de traitement de {TAL}.
Elle est traditionnellement implémentée par filtrage sur des chaînes de caractères ou des expressions rationnelles qui font référence à des termes du thème visé.
C'est regrettable car connaître la polarité d'une émotion est intéressant mais ça l'est plus encore si l'on sait à quoi le \textit{tweet} fait référence.

%Dans cette édition du DEFT, et contrairement à l'édition 2015 par exemple, les deux tâches mentionnées ci-dessus  ont été séparées.
D'un côté la tâche 1 consiste donc à décider si un \textit{tweet} traite ou non des transports et de l'autre côté la tâche 2 pour laquelle, étant donné un \textit{tweet} traitant des transports, il faut détecter la polarité: Neutre, Négatif, Positif ou Mixte.
L'article décrivant cette édition du Défi Fouille de Textes propose une description précise du processus de collecte et d'annotation du corpus~\cite{Paroubek-2018}, nous ne reviendrons donc pas plus ici sur les données d'entrée.
% Nous nous contenterons ici d'une présentation rapide du contenu des jeux de données d'apprentissage et de test et de l'influence que ceci a eu sur le développement de nos approches.

% Dans la section \ref{sec:corpus} nous décrirons un peu plus précisément le jeu de données puis
Dans la section~\ref{sec:methodes} nous décrirons les trois méthodes que nous avons mises en place ainsi qu'un système de vote entre les méthodes.
Ensuite, nous présenterons dans la section~\ref{sec:resultats} les résultats que nous avons obtenus dans la compétition officielle ainsi que des expériences plus complètes sur nos méthodes.
Enfin, nous proposerons quelques conclusions et perspectives sur ce travail dans la section~\ref{sec:conclusion}.

%%% Local Variables:
%%% mode: latex
%%% TeX-master: "../tweetaneuse"
%%% End:
