
 L'analyse automatique des \textit{tweets} est un domaine très actif du Traitement Automatique des Langues, notamment sur l'aspect étude de la polarité et détection des émotions.
 Ceci n'est pas très étonnant dans la mesure où les \textit{tweets}, et plus généralement les microblogs, constituent un moyen de jauger l'opinion individuelle d'une grande quantité de personnes.
 D'un autre côté, la classification binaire de \textit{tweets} selon qu'ils relèvent ou non d'une thématique particulière est une tâche connexe, souvent placé en amont de la chaîne de traitement de TAL. En effet, connaître la polarité d'une émotion est intéressant mais ça l'est plus encore si l'on sait à quoi le \textit{tweet} fait référence.
 Dans cette édition du DEFT, et contrairement à l'édition 2015 par exemple, les deux tâches ont été séparées. D'un côté la tâche 1 consistant à décider si un \textit{tweet} traite ou non des transports et de l'autre côté la tâche 2 pour laquelle, étant donné un \textit{tweet} traitant des transports, il faut détecter la polarité: Neutre, Négatif, Positif ou Mixte.

 Dans la section \ref{sec:corpus} nous décrirons un peu plus précisément le jeu de données puis dans la section \ref{sec:methodes} nous décrirons les trois méthodes que nous avons mises en place ainsi qu'un système de vote entre les méthodes.
 Ensuite, nous présenterons dans la section \ref{sec:resultats} les résultats que nous avons obtenus dans la compétition officielle ainsi que des expériences plus complètes sur nos méthodes.
 Enfin, nous proposerons quelques conclusions perspectives sur ce travail dans la section \ref{sec:conclusion}.
 
